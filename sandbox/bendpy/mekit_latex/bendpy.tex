\documentclass{marine_2015}
     
\usepackage{graphicx}
\usepackage{amsmath}
\usepackage{amsfonts}
\usepackage{amssymb}

\newcommand{\Amat}{\ensuremath{\mathbb{A}}}
\newcommand{\Bmat}{\ensuremath{\mathbb{B}}}
\newcommand{\bvec}{\ensuremath{\mathbf{b}}}
\newcommand{\uvec}{\ensuremath{\mathbf{u}}}
\newcommand{\Uvec}{\ensuremath{\mathbf{U}}}
\newcommand{\Pvec}{\ensuremath{\mathbf{P}}}



\newcommand{\bend}{\ensuremath{\text{Bend}\left|\text{P}\right|\text{y}}}
\newcommand{\RM}{\ensuremath{RM}}
\newcommand{\inner}[2]{\ensuremath{\left(#1, #2\right)}}
\newcommand{\rinner}[2]{\ensuremath{\left(#1, #2\right)_r}}
\newcommand{\tuple}[2]{\ensuremath{\left[#1, #2\right]}}

\newcommand{\ainner}[2]{\ensuremath{a\left(#1, #2\right)}}
\newcommand{\arinner}[2]{\ensuremath{a_r\left(#1, #2\right)}}
\newcommand{\binner}[2]{\ensuremath{b\left(#1, #2\right)}}
\newcommand{\brinner}[2]{\ensuremath{b\left(#1, #2\right)}}

\newcommand{\Linner}[1]{\ensuremath{L\left(#1\right)}}
\newcommand{\Vh}{\ensuremath{V_{\mathbf{n}}}}
\newcommand{\Qh}{\ensuremath{Q_{\mathbf{m}}}}

\newcommand{\norm}[1]{\ensuremath{\left\|#1\right\|}}
\newcommand{\vvec}[1]{\ensuremath{\pmb{#1}}}
\newcommand{\deriv}[2]{\ensuremath{\frac{\mathrm{d}#1}{\mathrm{d}#2}}}
\newcommand{\tderiv}[2]{\ensuremath{\tfrac{\mathrm{d}#1}{\mathrm{d}#2}}}

\title{
  BEND$\left|\text{P}\right|$Y: PYTHON FRAMEWORK FOR COMPUTING BENDING OF COMPLEX PLATE-BEAM SYSTEMS
}

\author{MIKAEL MORTENSEN$^{1, 2}$ AND MIROSLAV KUCHTA$^{1}$ AND KENT-ANDRE MARDAL$^{1, 2}$ }

\heading{Mikael Mortensen, Miroslav Kuchta and Kent-Andre Mardal}

\address{$^{1}$
Department of Mathematics, Division of Mechanics, University of Oslo,\\
0316 Oslo, Norway
  \and
$^{2}$ 
Center for Biomedical Computing, Simula Research Laboratory,\\
P.O. Box 134, No-134 Lysaker, Norway
}

\keywords{biharmonic equation, Lagrange multipliers, Schur complement}

\abstract{
We present a light-weight Python module for computing small deformations of a 
single plate supported by an arbitrary number of possibly intersecting
stiffeners. We show how the problem fits into the framework of abstract 
saddle point problems and how this abstraction can be exploited for clean design 
of the code. Stability properties of the resulting linear systems for two different 
sets of basis functions, namely, the eigenfunctions of the biharmonic operator 
and specialized Legendre polynomials are discussed. 
}

\begin{document}

\section{Introduction}
% What are we solving?
In this paper we discuss Galerkin methods for finding the equilibrium state of a 
physical system formed by a loaded thin plate and several beams which are constrained 
to deform together with the plate. Denoting $k$ the number of supporting beams, 
the equilibrium state is found as the solution of a constrained minimization problem
\begin{equation}
  \label{eq:foo}
  u = \min_{v\in V} \mathcal{E}\left(v\right)\quad\text{ and }\quad u_0\circ F_r
  = u_r, r=1, 2, \cdots k,
\end{equation}
where the energy functional $\mathcal{E}$ is defined for $u=\left(u_0, u_1,
\cdots, u_k\right)$ as
\[
  \mathcal{E}\left(u\right)=
    \frac{E_0}{2}\displaystyle\int_{\Omega}\Delta u_0\,\Delta u_0+
    \sum_{r=1}^k\frac{E_r}{2}\int_{\mathcal{I}}
  \deriv{^2u_r}{s^2}\deriv{^2u_r}{s^2}J_r^{-3}
  -\displaystyle\int_{\Omega}f u_0.
\]
Here $V=V_0\times V_1 \times\cdots\times V_k$ is a function space with $k+1$
components. The first component $V_0$ contains functions that map the plate
domain $\Omega$ to real numbers and is therefore the space where the plate's
vertical displacement $u_0$ is found. The remaining components in $V$ contain the 
vertical displacements of individual beams $u_r$, i.e., the scalar 
functions whose domain is the interval $\mathcal{I}$. Moreover each beam is
considered as a set $\Gamma_r=\left\{x\in\Omega, x=F_r\left(s\right), s\in\mathcal{I}\right\}$
defined by an invertible mapping $F_r$ with Jacobian $J_r$. Note that for straight 
beams which are in focus of the presented work the Jacobian is simply the length of
the beam divided by the length of the interval. The mapping $F_r$ is in addition 
required to satisfy the conditions $F_r\left(s_0\right), F_r\left(s_1\right)\in\partial\Omega$ 
for $s_0, s_1$ the endpoints of $\mathcal{I}$. No beam is thus allowed to end inside 
the plate. Finally $f$ denotes the load while $E_0, E_r, r=1,2,\cdots k$ are 
constant parameters which, following the Kirchhoff-Love and Euler-Bernoulli
hypothesis (e.g. \cite{reddy}), depend on the material and geometry.

Introducing $k$ Lagrange multipliers $\lambda_k\in Q_k$, where the functions in 
$Q_k$ map the interval $\mathcal{I}$ to scalars, the constrained problem 
(\ref{eq:foo}) can be equivalently written as a search for extrema of the Lagrangian
$\mathcal{L}$,
\begin{equation}
  \label{eq:bar}
\mathcal{L}\left(u, \lambda\right) = \mathcal{E}\left(u\right) +
  \sum_{r=1}^k\int_{\mathcal{I}}\left(u_0\circ F_r - u_r\right)\lambda_r J_r.
\end{equation}
The space $Q$ and function $\lambda\in Q$ are defined analogically as in (\ref{eq:foo}).
We note that the unconstrained problem (\ref{eq:bar}) is solvable only for
suitable pair of spaces $V, Q$ for which the requirements of the
Babu\v{s}ka-Brezzi\cite{babuska, brezzi} theory are satisfied.

% It can be solved with FEM but simple geometry ... global functions, spectral
% Galerkin
For a complex domain $\Omega$ the finite element method is arguably the most 
suitable method to solve the problem (\ref{eq:bar}). If, on the other hand, the
domain is simple, the Galerkin method with globally supported basis functions
can be applied. Note that the finite element method is an instance of the Galerkin 
method with test spaces spanned by the functions with local support. Regardless
of the basis functions employed, the stable discretization of (\ref{eq:bar})
requires that the Babu\v{s}ka-Brezzi conditions be satisfied on the constructed
finite dimensional spaces. We remark that in addition to stability
considerations the finite element discretization of problem is also complicated
by presence of the biharmonic operator which requires techniques for fourth order 
problems (see e.g. \cite{brenner}). The approximations of $V$ must be constructed 
from $C^1$ continuous elements, e.g. Argyris element \cite{argyris}, or non-conforming 
elements, e.g. Morley element\cite{morley}. Alternatively, discontinuous elements with 
suitable stabilization \cite{brenner_ip} can be used. 

Here we shall consider the plate as a simply connected rectangular domain and as
such focus on Galerkin methods with test functions having the global support.
Further, the two selected basis are designed for the fourth order problems and thus 
only the Babu\v{s}ka-Brezzi theory needs to be considered to derive a stable
discretization of problem (\ref{eq:bar}). The methods are discussed within a
framework for abstract saddle point problems reviewed in Section
\ref{sec:abstract}. The framework serves to identify the few common
elements(matrices) that are required to easily introduce Galerkin discretizations of
(\ref{eq:bar}) based on any set of basis functions. Properties of the two sets of
basis functions considered in this paper are then compared in Section \ref{sec:basis}.
Finally in Section \ref{sec:lbb} the inf-sup condition for the two proposed
disretizations is discussed.

\section{Abstract framework for saddle point problems}
\label{sec:abstract}
% conditions for extrema
The necessary conditions for the extreme point of the Lagrangian $\mathcal{L}$
from (\ref{eq:bar}) define $2k+1$ equations to be satisfied by the $2k+1$
unknowns $\left(u, \lambda\right)\in V\times Q$. These equations read
\[
  \begin{aligned}
    \label{eq:system}
    E_0\displaystyle\int_{\Omega}\Delta u_0\,\Delta v_0-
    \sum_{r=1}^k\int_{\mathcal{I}}v_0\lambda_r J_r &=\displaystyle\int_{\Omega}f
    v_0\quad\forall v_0\in V_0,& \\
    E_r\displaystyle\int_{\mathcal{I}}
    \deriv{^2u_r}{s^2}\deriv{^2v_r}{s^2}J_r^{-3} +
  \int_{\mathcal{I}} u_r \lambda_r J_r &= 0\quad\forall v_r\in V_r, r=1,
    2\cdots, k,&\\
  \int_{\mathcal{I}}\left(u_r-u_0\right)\mu_r J_r &= 0\quad\forall \mu_r\in Q_r,
    r=1, 2\cdots, k&.\\
  \end{aligned}
\]
System (\ref{eq:system}) fits into the abstract framework for saddle point
problems (see eq. Quarteroni \cite{quarteroni}). Within the framework, existence
and uniqueness of the solution of (\ref{eq:system}) can be discussed. Here we
shall assume the problem is indeed well-posed and instead use the abstractions
of the framework to identify the building blocks for efficient implementation
of the Galerkin method for the saddle point system. To simplify the notation we
let $\inner{\cdot}{\cdot}$ denote the $L^2$ inner product over the plate.
Moreover for $r=1, 2, \cdots, k$ the weighted $L^2$ inner product over the
interval $\mathcal{I}$ with the weight $J_r$ is denoted as $\rinner{\cdot}{\cdot}$.

The abstract saddle point problem is defined in terms of bilinear forms $a:V\times V\mapsto \mathbb{R}$,
$b:Q\times V\mapsto \mathbb{R}$ and a linear form $L:V\mapsto R$. To identify these 
forms in the plate-beam system (\ref{eq:system}), let $a_0:V_0\times V_0\mapsto
\mathbb{R}$ with $\ainner{u_0}{v_0}=E_0\inner{\Delta u_0}{\Delta v_0}$ and
$a_r:V_r\times V_r\mapsto \mathbb{R}$ such that $\arinner{u_r}{v_r}=E_r\rinner{J_r^{-2}\tderiv{^2 u_r}{s^2}}{J_r^{-2}\tderiv{^2
v_r}{s^2}}$ for $r=1, 2\cdots, k$. Moreover for we define $r$ bilinear forms
$b_r:Q_r\times V\mapsto\mathbb{R}$ by
$\brinner{\lambda_r}{u}=\rinner{u_0\circ F_r-u_r}{\lambda_r}$. It the follows 
that the bilinear forms $a$ and $b$ are simply
\[
  \ainner{u}{v} = \displaystyle\sum_{r=0}^{k}\arinner{u_r}{v_r}\quad\text{ and
  }\quad\binner{\lambda}{u} = \displaystyle\sum_{r=1}^{k}\brinner{\lambda_r}{u}.
\]
Finally with the linear form $\Linner{u}=\inner{f}{u_0}$ the problem
(\ref{eq:system}) is rewritten as an abstract saddle point problem: Find
$\left(u, \lambda\right)\in V\times Q$ such that
\begin{equation}
  \label{eq:abstract_saddle}
  \ainner{u}{v} + \binner{\lambda}{v} + \binner{\mu}{u} = \Linner{v}\quad\forall
  \left(v, \mu\right)\in V\times Q.
\end{equation}

To apply Galerkin method to system (\ref{eq:system}) we construct finite dimensional
subspaces $\Vh\subset V$ and $\Qh\subset Q$ that approximate their continuous
counter parts. As such the spaces $\Vh, \Qh$ have respectively $k+1$ and $k$
components with their individual dimensions
$\text{dim}\left(\Vh^r\right)=\mathbf{n}_r$, $\text{dim}\left(\Qh^r\right)=\mathbf{m}_r$ 
encoded in multi-indices $\mathbf{n}\in\mathbb{R}^{k+1}, \mathbf{m}\in\mathbb{R}^k$.
From this definition it follows that $n=\text{dim}\left(\Vh\right)$ and $m=\text{dim}\left(\Qh\right)$
are the sizes of the respected multi-indices. Considering the saddle point
problem (\ref{eq:system}) on the constructed subspaces yields a symmetric
indefinite linear system
\[
  \label{eq:sysAB}
    \begin{bmatrix}
      \mathbb{A} & \mathbb{B} \\
      \mathbb{B}^{\text{T}} & 0
    \end{bmatrix}
    \,
    \begin{bmatrix}
      \mathbf{U} \\
      \mathbf{P}
    \end{bmatrix}
    =
    \begin{bmatrix}
      \mathbf{b}\\
      0
    \end{bmatrix},
\]
for the unknown expansion coefficients $\Uvec\in\mathbb{R}^n,
\Pvec\in\mathbb{R}^m$ of the approximate solution
$\left(u_\mathbf{n}, \lambda_{\mathbf{m}}\right)\in \Vh \times \Qh$ of
(\ref{eq:system}).

Matrices $\Amat\in\mathbb{R}^{n\times n}$, $\Bmat\in\mathbb{R}^{n\times m}$ in
(\ref{eq:sysAB}) inherit the structure of the bilinear forms $a, b$. Specifically, 
the structure of $a$ translates into a block diagonal matrix $\Amat$ consisting of
$k+1$ blocks
\[
    \mathbb{A}=
    \begin{bmatrix}
      \mathbb{A}^0  &   &  &\\
                    & \mathbb{A}^1 &  &\\
                    &   &   \ddots    & \\
                    &   &   & \mathbb{A}^k\\
    \end{bmatrix},
\]
where the submatrices $\mathbb{A}^r$ are defined as $\mathbb{A}^r_{i,
j}=\arinner{\phi^r_i}{\phi^r_j}$ for $\phi^r_i, i=1, 2, \cdots \mathbf{n}_r$ the
basis functions of the component $\Vh^r$. Moreover, due to structure of $b$ the 
matrix $\Bmat$ consists of $k$ block columns
\[
    \mathbb{B}=
    \begin{bmatrix}
      \mathbb{C}^1 & \mathbb{C}^2 & \cdots & \cdots & \mathbb{C}^k\\
      \mathbb{M}^1   &        0       & \cdots & \cdots &        0      \\
           0         & \mathbb{M}^2   &    0   & \cdots &   \vdots      \\
         \vdots      &       0        & \ddots & \ddots &   \vdots      \\
         \vdots      &     \vdots      & \ddots & \ddots &       0      \\
      0         &  \cdots        & \cdots & 0      &   \mathbb{M}^k     \\
    \end{bmatrix}.
\]
Each column block consists of two sub-matrices
$\mathbb{C}^r\in\mathbb{R}^{n\times\mathbf{m}_r}$ and
$\mathbb{M}^r\in\mathbb{R}^{\mathbf{n}_r\times\mathbf{m}_r}$, which enforce the
constraint of equal deformation of the plate and $r$-th beam. The matrix
$\mathbb{C}^r$ with entries $\mathbb{C}^r_{i, j} = \rinner{\phi^0_i\circ F_r}{\psi^r_j}$ 
enforces the constraint on the plate, while matrix $\mathbb{M}^r$ defined by
$\mathbb{M}^r_{i, j}=-\rinner{\phi^r_i}{\psi^r_j}$ then enforces the constraint
on the beam. Note that the $\psi^r_j, j=1, 2,\cdots,\mathbf{m}_r$ denote the basis
functions of the Lagrange multiplier space $\Qh^r$.

It is clear from the structure of the matrices of the linear system
(\ref{eq:sysAB}) that its assembly requires three types of procedures (i) For 
every subspace $\Vh^r$ a matrix of discretizing the biharmonic operator is needed.
Note that for all the components but $\Vh^0$ the biharmonic operator is one 
dimensional. (ii). For every pair of spaces $\Vh^r, \Qh^r, r=1,2,\cdot k$ a mass
matrix $\mathbb{M}^r$ between the two spaces must be computed. (iii) Finally, for
every space $\Qh^r$ there must be a procedure for the computing matrix $\mathbb{C}^r$
which can be interpreted as a mass matrix between $\Vh^0$ and the space
$\Qh^r$. The number of procedures is thus significantly reduced if all the
spaces involved in the discrete formulation are spanned by the same functions. A
further step to speedy assembly is a choice of basis functions for which the
matrices $\mathbb{M}^r$ and $\mathbb{A}^r$ are sparse and the nonzero values can
be tabulated. With these choices computing the matrices $\mathbb{C}^r$ remains the
only bottleneck of the assembly process. 

% Some comments on python bmat, sympy.lambdify vectorize numpy leggauss.
In $\text{Bend}\!\left|\text{P}\right|\!\text{y}$ the integration of matrices $\mathbb{C}^r$ is built on top of
{\tt{NumPy}}\cite{numpy} and {\tt{SymPy}}\cite{sympy} modules. Specifically, using
{\tt{sympy.lambdify}} function, each integrand is turned from a symbolic representation 
to a lambda function which is vectorized for fast evaluation of {\tt{Numpy}}
arrays which represent the points of the Gauss-Legendre quadrature. The remaining
matrices of the linear system (\ref{eq:sysAB}) are sparse and efficiently stored
in the compressed sparsed row containers provided by {\tt{scipy.sparse}} module.
The two sets of basis functions which are currently supported by $\text{Bend}\!\left|\text{P}\right|\!\text{y}$
are discussed in the next section.

\section{Basis functions of the Galerkin method}
\label{sec:basis}
For sufficiently smooth solutions $\left(u, \lambda\right)$ of the saddle point
problem (\ref{eq:bar}) the integration by parts yields that the solution $u$ also
satisfies the system of partial differential equations
\[
  \begin{aligned}
    E_0\Delta^2u_0 &= f\quad\text{ in }\Omega\\
    E_r\deriv{^4 u_r}{s^4} &= 0\quad\text{ on }\mathcal{I}\\
    u \circ F_r&= u_r\quad\text{ on }\mathcal{I}\\
  \end{aligned}
\]
subjected to boundary conditions $u=0, \partial_nu=0$ on $\partial\Omega $and 
$u_i=0, \tderiv{u_i}{s}=0$ on $\partial\mathcal{I}$
or $u=0, \Delta u=0$ on $\partial\Omega$ and $u_i=0, \tderiv{^2u_i}{s^2}=0$ on
$\partial\mathcal{I}$. The first set of boundary conditions is known as clamped as
it fixes both the displacement and the rotation of the system on the boundary. With
the latter, simply supported boundary conditions, only the displacement is fixed. 
Both boundary conditions are supported in $\text{Bend}\!\left|\text{P}\right|\!\text{y}$. 

The clamped boundary conditions are enforced onto the solution by constructing
the test spaces of the Galerkin method from functions $S_i$ presented in Shen
\cite{shen_paper}. Functions $S_i$ defined over the interval
$\mathcal{I}=\left[-1, 1\right]$ are linear combinations of Legendre
polynomials where the coefficients of the combinations are chosen such that
$S_i=0, \tderiv{S_i}{s}=0$ in the endpoints of the domain. To span an $n$
dimensional space $\Vh^r, r>0$ defined over the interval $\mathcal{I}$, the first 
$n$ functions $S_i$ are taken as the basis. For $n^2$ dimensional space $\Vh^0$ 
defined over $\Omega=\mathcal{I}\times\mathcal{I}$ we use as the basis functions 
products of the first $n$ functions in each Cartesian direction, that is, the basis 
functions have the form $S_i\left(x\right)S_j\left(y\right)$. Further the choice of
Shen basis functions yields matrices the one dimensional bending matrix
$\mathbb{A}^r$ as identity while the mass matrix $\mathbb{M}^r$ is sparse and
pentadiagonal. We refer reader to the original paper \cite{shen_paper} for the
tabulated non-zero entries of the matrix. Finally the matrix the matrix of the two
dimensional biharmonic operator $\mathbb{A}\in\mathbb{R}^{n^2\times n^2}$ is obtained
as a Kronecker product of one dimensional mass matrix $\mathbb{M}\in\mathbb{R}^n$ and 
the stiffness matrix $\mathbb{C}\in\mathbb{R}^n$, i.e. $\mathbb{A}^0 = \mathbb{M}\otimes\mathbb{I} + 
2\mathbb{C}\otimes\mathbb{C} + \mathbb{I}\otimes\mathbb{M}$. We note that the
condition number of matrix $\mathbb{A}^0$ grows exponentially with $n$.

%sines
To enforce simply supported boundary conditions onto solution of (\ref{eq:bar}) 
we consider test spaces constructed from functions $E_k=\sqrt{\tfrac{2}{\pi}}\sin{kx}$,
which are eigenfunctions of the eigenvalue problem for one dimensional biharmonic 
operator
\[
  \begin{aligned}
  \deriv{^4 u}{s^4} = \lambda u\text{ in }\mathcal{I},\\
  u = \deriv{^2 u}{s^2} = 0\text{ on }\partial\mathcal{I}
  \end{aligned}
\]
with the interval $\mathcal{I}=\left[0, \pi\right]$. The corresponding
eigenvalues are $\lambda_k=k^4$. In complete analogy to the spaces spanned by
functions due to Shen, the basis of the $n$-dimensional space $\Vh^r, r>0$ over
$\mathcal{I}$ is formed by first $n$ eigenfunctions $E_k$. The basis of the
space $\Vh^0$ defined over $\Omega=\mathcal{I}\times\mathcal{I}$ are then formed
as tensor products. Attractive property of the eigenfunctions $E_k$ is the fact
that all the matrices $\mathbb{A}^r$, $\mathbb{M}^r$ are diagonal. Indeed we
have $\mathbb{M}^r=\mathbb{I}$, while $\mathbb{A}^r\in\mathbb{R}^{n \times n},
r>0$ takes the form $\mathbb{A}^r=\text{diag}\left(\lambda_k, k=1, 2, \cdots
n\right)$. Finally the two dimensional bending matrix is defined as $\mathbb{A}^0_{ij}=
\Lambda_{ij}\delta_{ij}$ with the diagonal values $\Lambda_{ij}= i^4 + 2 i^2 j^2 + j^4$.
It is clear that the condition number of the matrix grows as $n^4$. 

%comparison
We shall now compare the convergence properties of the two basis on a simple one
dimensional biharmonic problem
\begin{equation}
  \label{eq:test}
  \deriv{^4u}{s^4} = f\text{ in }\left(-1, 1 \right)\quad\text{ with }f(x)=\begin{cases}
      g(x) & x \leq 0 \\
      h(x) & x > 0 \\
    \end{cases},
  \end{equation}
which can be viewed as a building block of the complex-plate beam system (\ref{eq:foo}).
We remark that the Shen basis solves the problem (\ref{eq:test}) with clamped
boundary conditions, while for the basis of eigenfunction simply supported
boundary conditions are used. The problem is considered with four different right 
hand sides $f_i$. It can be seen that each function $f_i$ is determined by two functions, $g, h$. This
property is symbolically denoted as $f_i=\left(g, h\right)$. With this
convention, the right hand sides considered are $f_0=\inner{1}{2}$, $f_1=\inner{f}{x+1}$,
$f_2=\inner{x^2/2 + x/4 + 3/4}{h=-x^2 + x/4 + 3/4}$ and
$f_3=\inner{\exp{x}\sin{5x}}{\exp{x}\sin{5x}}$. The functions $f_i$ are
respectively $L^2, H^1, H^2$ and $C^{\infty}$ smooth. The smoothness of the
solution $u$ of (\ref{eq:test}) is then four degrees higher than the given
right-hand side. As such, in all but the last case, fixed convergence is to be
expected from the Galerkin method with Shen basis functions. In the last case
the method should converge exponentially. These expectations are confirmed by
the results shown in Table \ref{tab:shen_convergence}. On the other hand simple error
estimate for the Galerkin method with eigenfunctions $E_k$ shows that the $L^2$
norm of the error is bounded as $\norm{e}_0\leq\tfrac{\norm{f}_0}{n^4}$. The
error of the solution should therefore decrease no slower than quadratically.
This estimate is confirmed by the results listed in Table
\ref{tab:sine_convergence}.
\begin{table}[t!]
    \begin{center}
    \begin{tabular}{ccccc}
\hline
$n$  &  $f_0: \norm{e}_0$  & $f_1: \norm{e}_0$ & $f_2: \norm{e}_0$ & $f_3: \norm{e}_0$\\
\hline
8    & 6.1251E-06(4.11) & 5.7545E-06(4.11) & 2.8982E-08(5.84)& 6.2305E-06(4.49) \\
16   & 2.9912E-07(4.36) & 2.8123E-07(4.35) & 4.1493E-10(6.13)& 3.0082E-07(4.37) \\
32   & 1.3929E-08(4.42) & 1.4545E-08(4.27) & 4.8407E-12(6.42)& 1.4161E-08(4.41) \\
64   & 6.2978E-10(4.47) & 5.9356E-10(4.61) & 5.6482E-14(6.42)& 6.5283E-10(4.44) \\
128  & 2.5395E-11(4.63) & 2.4049E-11(4.63) & 8.5939E-16(6.04)& 2.7228E-11(4.58) \\
256  & 1.1403E-12(4.48) & 1.0861E-12(4.47) & 7.0351E-16(0.29)& 1.2692E-12(4.42) \\
\hline
\hline
    \end{tabular}
    \caption{The error and convergence rate of the Galerkin method with eigenfunctions for
    problem (\ref{eq:test}) with simply supported boundary conditions.
  Irrespective of the regularity of the right-hand sides $f_i$ the method
converges with fixed order. In agreement with the theoretical analysis, the
order is greater then four.}
  \label{tab:sine_convergence}
  \end{center}
  \end{table}

  The results presented thus far have validated the Galerkin method with basis
  of Shen functions and eigenfunctions as viable methods to solve the biharmonic
  equations. Interpreting these as subproblems in the system (\ref{eq:bar}) we
  claim that the methods are suitable also to tackle the complex plate-beam system.
  At the time of writing we are unable to provide a support for this claim in
  the form of a convergence study. However Figure \ref{fig:solutions} shows that
  the solutions obtained by the method are physical and the constrains remain
  well respected.

\begin{table}[t!]
    \begin{center}
    \begin{tabular}{ccccc}
\hline
$n$  &  $f_0: \norm{e}_0$  & $f_1: \norm{e}_0$ & $f_2: \norm{e}_0$ & $f_3: \norm{e}_0$\\
\hline
4 &  3.0367E-05(2.34)& 1.1478E-05(4.11)& 7.5138E-07(8.28)& 8.7249E-05(2.65)  \\ 
6 &  9.4639E-06(2.88)& 2.0211E-06(4.28)& 1.1883E-07(4.55)& 1.1292E-05(5.04) \\
8 &  3.7833E-06(3.19)& 5.5737E-07(4.48)& 2.7564E-08(5.08)& 7.2035E-07(9.57) \\
10&  1.7737E-06(3.39)& 1.9868E-07(4.62)& 1.0879E-08(4.17)& 2.8851E-08(14.42) \\
12&  9.2954E-07(3.54)& 8.3836E-08(4.73)& 3.5788E-09(6.10)& 4.2572E-10(23.12) \\
14&  5.2899E-07(3.66)& 3.9893E-08(4.82)& 1.1724E-09(7.24)& 8.7160E-12(25.23) \\
16&  3.2080E-07(3.75)& 2.0775E-08(4.89)& 6.8783E-10(3.99)& 4.7527E-14(39.03) \\
18&  2.0463E-07(3.82)& 1.1004E-08(5.40)& 4.3562E-10(3.88)& 8.3310E-15(14.78) \\
20&  1.3602E-07(3.88)& 5.4562E-09(6.66)& 2.5669E-10(5.02)& 1.7601E-14(-7.10) \\
22&  9.3564E-08(3.93)& 3.2351E-09(5.48)& 1.5121E-10(5.55)& 2.0605E-13(-25.81)\\
24&  6.3166E-08(4.52)& 2.3836E-09(3.51)& 7.8326E-11(7.56)& 2.8589E-13(-3.76) \\
26&  4.8081E-08(3.41)& 1.9361E-09(2.60)& 4.5928E-11(6.67)& 1.0247E-12(-15.95)\\
28&  3.2319E-08(5.36)& 1.2840E-09(5.54)& 3.0818E-11(5.38)& 1.9021E-12(-8.35) \\
30&  2.0739E-08(6.43)& 9.6124E-10(4.20)& 1.7277E-11(8.39)& 2.7543E-11(-38.74)\\
\hline
    \end{tabular}
    \caption{
    The error and convergence rate of the Galerkin method with Shen basis for
    problem (\ref{eq:test}) with clamped boundary conditions. Fixed order
    convergence for the first three right hand sides and exponential convergence
    for $f_3$ confirm that the convergence rate is determined by the regularity of 
    the solution.
    }
  \label{tab:shen_convergence}
  \end{center}
  \end{table}

 \begin{figure}[ht]
 \centering
 \includegraphics[width=0.45\textwidth]{img/shen_u0}
 \includegraphics[width=0.45\textwidth]{img/sine_u0}\\
 \includegraphics[width=0.45\textwidth]{img/shen_u0_ur}
 \includegraphics[width=0.45\textwidth]{img/sine_u0_ur}\\
 %\caption
 \caption{
  Solution of the plate-beam system obtained by the Galerkin method with Shen
  functions (left) and eigen functions (right) for $n=15$. In the top row the vertical
  displacement $u_0$ of the plate is plotted. The position of beams is indicated
  by black lines. Further the index of the beam is shown in magenta. In the
  bottom row the difference in the plate's displacement on the beams and the
  beams' displacement is shown. The error of constrains is less than $10^{-4}$.
 }
 \label{fig:solutions}
 \end{figure}

\section{Condition numbers of plate-beam systems and preconditioning}
\label{sec:lbb}
In this section we briefly discuss a phenomena which is bound to occur in any
discretization of plate-beam system, namely, the condition number of linear system 
grows with the size of the linear system (see e.g.
\cite{benzi}). This growth is due to the saddle point nature of the continuous
problem (\ref{eq:system}) but it can be greatly accelerated by the choice of the
discretization. To illustrate this issue Figure \ref{fig:no_precon} shows the condition 
number of linear systems obtained by discretizing (\ref{eq:system}) with the
Galerkin method with Shen polynomials and the eigenfunctions. The condition
number of the system due to Shen basis reaches $10^{16}$ very rapidly ($n=15$). For 
same degree the condition number of the eigenfunction system is about ten orders of 
magnitude smaller. We note that $n=15$ is a modest degree and to achieve
sufficient accuracy of the solution, spaces with larger dimensions might be
needed. However with larger spaces the linear systems quickly become stiff and
the obtained solution might suffer from round-off errors. This problem can be
avoided by solving a preconditioned problem. Here we shall hint on a part of the
suitable preconditioner. Note that for simplicity we set $E_r=1$ for all $r$.

Investigating the spectra of linear systems in Figure \ref{fig:no_precon} it is
clear that the growth of the condition number is mainly due to the fact that the
smallest eigenvalue of the system decreases with the size of the linear system.
Using results of \cite{marie} that build on the work of \cite{malkus}, the
smallest eigenvalue defines the discrete Brezzi inf-sup constant $\beta$, which
determines stability of the discretization. In \cite{marie} stability of the
finite element disretization of the Stokes(saddle point) system is investigated using
the eigenvalue problem for the Schur complement
\begin{equation}
  \label{eq:schur}
  \Bmat^{\text{T}}\Amat^{-1}\Bmat\Pvec = \beta\mathbb{N}\Pvec,
\end{equation}
where $\mathbb{N}$ is a symmetric positive definite matrix which
represent discretization of the norm of the continuous space $Q$. The matrix is
\textit{a priori} known and the discretization is found stable if the smallest 
of eigenvalues $\beta$ is bounded regardless of the size of the linear system. Here
we shall reverse the process: the matrix $\mathbb{N}$ is to be found from the
requirement that the smallest eigenvalue remains bounded. It can be shown that
such a matrix can be found by the following algorithm. Let $\mathbb{M},
\mathbb{A}$ the mass matrix and the matrix of the biharmonic operator on $\Qh$.
Then
$\mathbb{N}=\left(\mathbb{M}\mathbb{V}\right){\Lambda}^{-1}\left(\mathbb{M}\mathbb{V}\right)^{\text{T}}$,
where $\mathbb{V}$, $\Lambda$  solve the generalized eigenvalue problem
$\mathbb{A}\mathbb{V}=\mathbb{M}\mathbb{V}\Lambda$. We note that with the basis
of eigenfunctions we have $\mathbb{V}=\mathbb{I}$ and $\Lambda=\mathbb{A}$. The
smallest eigenvalue of the problem (\ref{eq:schur}) with the constructed matrix
$\mathbb{N}$ is shown in Figure \ref{fig:schur}. The eigenvalue remains constant
for all values of discretization parameter $n$.
 \begin{figure}[ht]
 \centering
 \includegraphics[width=0.45\textwidth]{img/shen_cond}
 \includegraphics[width=0.45\textwidth]{img/sine_cond}\\
 \includegraphics[width=0.45\textwidth]{img/shen_spectrum}
 \includegraphics[width=0.45\textwidth]{img/sine_spectrum}\\
 \caption{Condition number and spectra of linear systems stemming from
 discretization of the plate-beam system (\ref{eq:bar}) by Shen polynomials
 (left) and eigenfunctions (right) for different degrees $n$. Two beam arrangements 
 are considered. Subscripts one corresponds to a single vertical beam
 $\left[0, -1\right]-\left[0, 1\right]$. Arrangement where a horizontal beam 
$\left[-1, 0\right]-\left[1, 0\right]$ is added is denoted by subscript two.
For both discretizations the condition numbers grows roughly as
$\kappa\propto\exp{n^{1/\alpha}}$, where $\alpha$ for the Shen basis is much
larger than for the eigenfunctions. The presence of the second beam has little to
no effect on the rate. In the bottom row the dependence of the smallest and largest
eigenvalues $\lambda_{min}, \lambda_{max}$ of the system on the degree is shown. 
The largest eigenvalues are stable, while the smallest ones go zero and are this 
the cause of the growth of the condition number $\kappa$.
 }
 \label{fig:no_precon}
 \end{figure}
 
 \begin{figure}[ht]
 \centering
 \includegraphics[width=0.45\textwidth]{img/Schur_precond_shen_cond}
 \includegraphics[width=0.45\textwidth]{img/Schur_precond_sine_cond}\\
 \caption{Smallest eigenvalues of the problem (\ref{eq:schur}) with constructed
 matrix $\mathbb{N}$. On the left the system was discretized with function due
 to Shen. Result of the system discretized with eigenfunctions are shown on the
 right. The smallest eigenvalue $\beta_1, \beta_2$ for the system with one and
 two beams remain constant for all $n$.}
 \label{fig:Schur}
 \end{figure}
Based on our findings and the theory of abstract preconditioner reviewed in
\cite{kent} a possible preconditioner for the plate-beam system is a block
diagonal matrix
\[
    \begin{bmatrix}
      \mathbb{A}^{-1} & 0 \\
      0 & \mathbb{N}^{-1}
    \end{bmatrix}.
\]
The effect of the proposed preconditioner is shown in Figure \ref{eq:precond}.
The preconditioner has stabilized the smallest eigenvalues but at the same time 
linear growth has been introduced to the largest eigenvalues. This growth translates 
into a linear increase of the condition number of the system. This is certainly an 
improvement over the unpreconditionerd case. However the ideal preconditioner
would yield stable condition numbers.

 \begin{figure}[ht]
 \centering
 \includegraphics[width=0.45\textwidth]{img/Precond_shen_cond}
 \includegraphics[width=0.45\textwidth]{img/Precond_sine_cond}\\
 \includegraphics[width=0.45\textwidth]{img/prec_shen_spectrum}
 \includegraphics[width=0.45\textwidth]{img/prec_sine_spectrum}\\
 \caption{
 Condition number and spectra of the preconditioned linear systems stemming from
 discretization of the plate-beam system (\ref{eq:bar}) by Shen polynomials
 (left) and eigenfunctions (right) for different degrees $n$. Two beam arrangements 
 are considered. Subscripts one corresponds to a single vertical beam
 $\left[0, -1\right]-\left[0, 1\right]$. Arrangement where a horizontal beam 
$\left[-1, 0\right]-\left[1, 0\right]$ is added is denoted by subscript two.
For both discretizations the condition numbers grows linearly with $n$. The
growth is due the growth of the largest eigenvalues of the systems}
 \label{fig:precond}
 \end{figure}


\section{Conclusions}
\label{sec:end}
fff
%%%%%%
% FIXME references
% FIXME typos
% FIXME consistency
%%%%%%

\newpage
\begin{thebibliography}{99}

\bibitem{reddy} Reddy B.D., \textit{Introductory functional analysis: with applications
  to boundary value problems and finite elements}, Springer, (1998).

%\bibitem{brezzi} Brezzi F., On the existence, uniqueness and approximation of saddle-point problems arising from Lagrangian multipliers,
%  \textit{Mathematical Modelling and Numerical Analysis-Mod{\'e}lisation Math{\'e}matique et Analyse Num{\'e}rique},
%  (1974), 
%  \textbf{8}: 129--151.
%
%\bibitem{brenner} Brenner S.C. and Scott R., \textit{The Mathematical Theory of Finite Element Methods},
%  Springer, (2008).
%
%\bibitem{babuska} Babu\v{s}ka I., The finite element method with Lagrangian
%  multipliers, \textit{Numerische Mathematik}, (1973), \textbf{20(3)}: 179--192.
%
%\bibitem{sympy} {{SymPy Development Team}},
%  \textit{SymPy: Python library for symbolic mathematics}, (2014),
%  http://www.sympy.org.
%
%\bibitem{shen} Shen, J., Efficient spectral-Galerkin method I. Direct solvers of
%  second-and fourth-order equations using Legendre polynomials, \textit{SIAM
%  Journal on Scientific Computing}, (1994), \textbf{15(6)}: 1489--1505.
%
  % \bibitem{numpy}
  %   @Misc{,
  %   author =    {Eric Jones and Travis Oliphant and Pearu Peterson and others},
  %   title =     {{SciPy}: Open source scientific tools for {Python}},
  %   year =      {2001--},
  %   url = "http://www.scipy.org/",
  %   note = {[Online; accessed 2015-05-03]}
  % }
  % 
  % \bibitem{cython}
  % Stefan Behnel, Robert Bradshaw, Craig Citro, Lisandro Dalcin, Dag Sverre Seljebotn and Kurt Smith. Cython: The Best of Both Worlds, Computing in Science and Engineering, 13, 31-39 (2011),
  % 
  % 
  % \bibitem{fenics}
  %   Logg A., Mardal K.-A., Wells G.N. et al.,
  %   \textit{Automated Solution of Differential Equations by the Finite Element Method},
  %   Springer, (2012).
  % 
  % \bibitem{swig}
  %   @inproceedings{Beazley:1996:SEU:1267498.1267513,
  %  author = {Beazley, David M.},
  %  title = {SWIG: An Easy to Use Tool for Integrating Scripting Languages with C and C++},
  %  booktitle = {Proceedings of the 4th Conference on USENIX Tcl/Tk Workshop, 1996 - Volume 4},
  %  series = {TCLTK'96},
  %  year = {1996},
  %  location = {Monterey, California},
  %  pages = {15--15},
  %  numpages = {1},
  %  url = {http://dl.acm.org/citation.cfm?id=1267498.1267513},
  %  acmid = {1267513},
  %  publisher = {USENIX Association},
  %  address = {Berkeley, CA, USA},
  % } 
  % 
  % \bibitem{oasis}
  % @article{mortensen2015oasis,
  %   title={Oasis: A high-level/high-performance open source Navier--Stokes solver},
  %   author={Mortensen, Mikael and Valen-Sendstad, Kristian},
  %   journal={Computer Physics Communications},
  %   volume={188},
  %   pages={177--188},
  %   year={2015},
  %   publisher={North-Holland}
  % }

\end{thebibliography}
\end{document}
