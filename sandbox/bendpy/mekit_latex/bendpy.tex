\documentclass{marine_2015}
     
\usepackage{graphicx}
\usepackage{amsmath}
\usepackage{amsfonts}
\usepackage{amssymb}

\newcommand{\Amat}{\ensuremath{\mathbb{A}}}
\newcommand{\Bmat}{\ensuremath{\mathbb{B}}}
\newcommand{\bvec}{\ensuremath{\mathbf{b}}}
\newcommand{\uvec}{\ensuremath{\mathbf{u}}}
\newcommand{\RM}{\ensuremath{RM}}
\newcommand{\inner}[2]{\ensuremath{\left(#1, #2\right)}}
\newcommand{\Vh}{\ensuremath{\mathcal{V}_h}}
\newcommand{\tuple}[2]{\ensuremath{\left[#1, #2\right]}}

\newcommand{\ainner}[2]{\ensuremath{a\left(#1, #2\right)}}
\newcommand{\binner}[2]{\ensuremath{b\left(#1, #2\right)}}
\newcommand{\Linner}[1]{\ensuremath{L\left(#1\right)}}

\newcommand{\norm}[1]{\ensuremath{\left\|#1\right\|}}
\newcommand{\vvec}[1]{\ensuremath{\pmb{#1}}}
\newcommand{\deriv}[2]{\ensuremath{\frac{\mathrm{d}#1}{\mathrm{d}#2}}}
\newcommand{\tderiv}[2]{\ensuremath{\tfrac{\mathrm{d}#1}{\mathrm{d}#2}}}

% FIXME remove
\usepackage{xcolor}

\title{
  BEND$\left(\text{P}\right|$Y: PYTHON FRAMEWORK FOR COMPUTING BENDING OF COMPLEX PLATE-BEAM SYSTEMS
}

\author{MIKAEL MORTENSEN$^{1, 2}$ AND MIROSLAV KUCHTA$^{1}$ AND KENT-ANDRE MARDAL$^{1, 2}$ }

\heading{Mikael Mortensen, Miroslav Kuchta and Kent-Andre Mardal}

\address{$^{1}$
Department of Mathematics, Division of Mechanics, University of Oslo,\\
0316 Oslo, Norway
  \and
$^{2}$ 
Center for Biomedical Computing, Simula Research Laboratory,\\
P.O. Box 134, No-134 Lysaker, Norway
}

\keywords{TODO}

\abstract{
We present a light-weight Python module for computing small deformations of a 
single plate supported by an arbitrary number of possibly intersecting stiffeners,
whose shape need not to be linear. We show how the problem fits into the
framework of abstract saddle point problems and how this abstraction can be 
exploited for clean design of the code. We discuss properties of the resulting 
linear system for two different sets of basis functions, namely, the eigenfunctions
of the biharmonic operator and specialized Legendre polynomials.
}

\begin{document}

\section{Introduction}
Some ideas that could go here
\begin{itemize}
  \item Solving biharmonic with FEM requires $C^1$ elements, not so easy.
    Therefore mixed formulations are more common but this leads to larger
    systems.
  \item Why Python - suitable for large scale (Oasis, spectralDNS, etc) but ease
    of implementation together with abundance of libraries (sympy, numpy) make
    it a great tool for exploration. This is what BendPy does.
  \item Question that we are asking is which of basis is better == numerically
    stable + convergence. Simple domains. Indication of suitable preconditioner.
  \item outline
\end{itemize}

\section{The math}
Let $\mathcal{I}$ denote a nonempty interval. As we are in the following
interested in domains that have a Cartesian product structure we consider without
loss of generality $\Omega=\mathcal{I}\times\mathcal{I}$. We shall refer to this
domain as \textit{plate}. Further let $w_i, i=1, 2,\cdots, k$ be a set of curves 
$w_i=\left\{\vec{x}\in\Omega, \vec{x}=\vec{F}_i\left(s\right),
s\in\mathcal{I}\right\}$, where $\vec{F_i}$ is a smooth invertible mapping with
Jacobian $J_i>0$ that additionally satisfies $\vec{F}_i\left(s_0\right),
\vec{F}_i\left(s_1\right)\in\partial\Omega$ and
$\vec{F}_i\left(s_0\right) \neq \vec{F}_i\left(s_1\right)$ for $s_0, s_1$ the
endpoints of interval $\mathcal{I}$. These curves, and occasionally the
respected mappings are referred to as \textit{beams}. In this paper we are for the
most part concerned with straight/linear beams, that is, we consider mappings
$\vec{F}_i\left(s\right)=\vec{P}_i\tfrac{s_1-s}{s_1-s_0}+\vec{Q}_i\tfrac{s_0-s}{s_0-s_1}$
determined by pairs of mutually distinct points $\vec{P}_i, \vec{Q}_i$ which
are located on the boundary of the plate. Note that in this case
$J_i=\tfrac{\left|\vec{P}_i-\vec{Q}_i\right|}{2}$.

With these assumptions on geometry we let $V, \hat{V}_i, i=1, 2, \cdots, k$ denote
spaces of functions that map respectively the plate and the beams to real
numbers. By invertibility of $\vec{F}_i$ each function space $\hat{V}_i$ can be
associated with a function space that $V_{i}$ that maps the reference interval
$\mathcal{I}$ to real numbers. Indeed for $v\in V_i$ function
$\hat{v}=v\circ\vec{F}_i$ belongs to $\hat{V}_i$.

Let $u\in V, u_i\in V_i, i=1, 2, \cdots, k$ and consider the problem of
minimizing the Lagrangian
\[
  \mathcal{L}\left(u, u_1, u_2, \cdots, u_k\right)=
  \frac{E}{2}\displaystyle\int_{\Omega}\Delta u\,\Delta u+
  \sum_i\frac{E_i}{2}\int_{\mathcal{I}}
  \deriv{^2u_i}{s^2}\deriv{^2u_i}{s^2}J_i^3
  -\displaystyle\int_{\Omega}f u
\]
subjected to $k$ constraints
$T(u)=u_i$, here $T(u)$ is a short hand for the composition of trace and the
pullback inverse.\textit{explain the constraint, trace, Sobolev?}.
We build the constraint into the Lagrangian. To this end consider spaces $Q_i$
and functions (Lagrange multipliers) $\lambda_i\in Q_i$. Moreover let
$Q=Q_0\times Q_1\times\cdots\times Q_k$ and $\vvec{\lambda}\in
Q,\vvec{\lambda}_i=\lambda_i$ and a problem
\[
  \mathcal{L}\left(u, u_1, u_2, \cdots, u_k;\vvec{\lambda}\right)=
   \frac{E}{2}\displaystyle\int_{\Omega}\Delta u\,\Delta u+
  \sum_i\frac{E_i}{2}\int_{\mathcal{I}} \deriv{^2u_i}{s^2}\deriv{^2u_i}{s^2}J_i^3
  -\displaystyle\int_{\Omega}f u - \sum_i\int_{\mathcal{I}}\left(u-u_i\right)\lambda_i J_i
\]
The necessary condition for extrema are then
\[
  \begin{aligned}
    E\displaystyle\int_{\Omega}\Delta u\,\Delta v- \sum_i\int_{\mathcal{I}}v\lambda_i J_i
    &=\displaystyle\int_{\Omega}f v\quad\forall v\in V& \\
  E_i\displaystyle\int_{\mathcal{I}} \deriv{^2u_i}{s^2}\deriv{^2v_i}{s^2}J_i^3 +
  \int_{\mathcal{I}} u_i \lambda_i J_i &= 0\quad\forall v_i\in V_i&\\
  \int_{\mathcal{I}}\left(u_i-u\right)\mu_i J_i &= 0\quad\forall \mu_i\in Q_i&\\
  \end{aligned}
\]
Under additional regularity we can make Euler Lagrange equations
\[
  \begin{aligned}
    E\Delta^2u &= f\quad\text{ in }\Omega\\
    E_i\deriv{^4 u_i}{s^4} &= 0\quad\text{ on }\mathcal{I}\\
    u &= u_i\quad\text{ on }\mathcal{I}\\
  \end{aligned}
\]
subjected to boundary conditions $u=0, \partial_nu=0$ and $u_i=0, \tderiv{u_i}{s}=0$
$u=0, \Delta u=0$ and $u_i=0, \tderiv{^2u_i}{s^2}=0$. \textit{What are these called?}
Abstract saddle $V:=V\times V_1, \times V_2, \cdots, \times V_k$,
$V\ni\vvec{u}=\left(u, u_1, u_2, \cdots, u_k\right)$ and define bilinear form
$a:V\times V\mapsto\mathbb{R}$ as
\[
  \ainner{\vvec{u}}{\vvec{v}} = 
  E\displaystyle\int_{\Omega}\Delta u\,\Delta+
  \sum_iE_i\displaystyle\int_{\mathcal{I}} \deriv{^2u_i}{s^2}\deriv{^2v_i}{s^2}J_i^3
\]
$b:V\times Q\mapsto\mathbb{R}$ as
\[
  \binner{\vvec{v}}{\vvec{\lambda}} = 
  \int_{\mathcal{I}}\left(v_i-v\right)\lambda_i J_i
\]
Finally linear form
$L:V\mapsto\mathbb{R}$ as
\[
  \displaystyle\int_{\Omega}f v
\]
%FIXME it is much better to use V_0 and then V...
% Saddle
Then the problem becomes simply: Find $\vvec{u}\in V, \vvec{\lambda}\in Q$ such
that $\ainner{u}{v}+\binner{v}{\lambda}+\binner{u}{\mu}=\Linner{v}$ for all 
$\vvec{v}\in V, \vvec{\mu}\in Q$. You have babuska theory that gives you
continuous existence. Don't go there... Discrete mention Stokes taylor hood, or
the compatibility condition from spectral.
% Galerkin

% Matrices

% Discussion on properties of basis. Some speedup with FFT

% Eigenvalue problems related to LBB, do both?

% Preconditioning

\end{document}
