\documentclass{marine_2015}
     
\usepackage{graphicx}
\usepackage{amsmath}
\usepackage{amsfonts}
\usepackage{amssymb}

\newcommand{\Amat}{\ensuremath{\mathbb{A}}}
\newcommand{\Bmat}{\ensuremath{\mathbb{B}}}
\newcommand{\bvec}{\ensuremath{\mathbf{b}}}
\newcommand{\uvec}{\ensuremath{\mathbf{u}}}
\newcommand{\RM}{\ensuremath{RM}}
\newcommand{\inner}[2]{\ensuremath{\left(#1, #2\right)}}
\newcommand{\tuple}[2]{\ensuremath{\left[#1, #2\right]}}

\newcommand{\ainner}[2]{\ensuremath{a\left(#1, #2\right)}}
\newcommand{\binner}[2]{\ensuremath{b\left(#1, #2\right)}}
\newcommand{\Linner}[1]{\ensuremath{L\left(#1\right)}}
\newcommand{\Vh}{\ensuremath{V_{\mathbf{n}}}}
\newcommand{\Qh}{\ensuremath{Q_{\mathbf{m}}}}

\newcommand{\norm}[1]{\ensuremath{\left\|#1\right\|}}
\newcommand{\vvec}[1]{\ensuremath{\pmb{#1}}}
\newcommand{\deriv}[2]{\ensuremath{\frac{\mathrm{d}#1}{\mathrm{d}#2}}}
\newcommand{\tderiv}[2]{\ensuremath{\tfrac{\mathrm{d}#1}{\mathrm{d}#2}}}

% FIXME remove
\usepackage{xcolor}

\title{
  BEND$\left(\text{P}\right|$Y: PYTHON FRAMEWORK FOR COMPUTING BENDING OF COMPLEX PLATE-BEAM SYSTEMS
}

\author{MIKAEL MORTENSEN$^{1, 2}$ AND MIROSLAV KUCHTA$^{1}$ AND KENT-ANDRE MARDAL$^{1, 2}$ }

\heading{Mikael Mortensen, Miroslav Kuchta and Kent-Andre Mardal}

\address{$^{1}$
Department of Mathematics, Division of Mechanics, University of Oslo,\\
0316 Oslo, Norway
  \and
$^{2}$ 
Center for Biomedical Computing, Simula Research Laboratory,\\
P.O. Box 134, No-134 Lysaker, Norway
}

\keywords{TODO}

\abstract{
We present a light-weight Python module for computing small deformations of a 
single plate supported by an arbitrary number of possibly intersecting stiffeners,
whose shape need not to be linear. We show how the problem fits into the
framework of abstract saddle point problems and how this abstraction can be 
exploited for clean design of the code. We discuss properties of the resulting 
linear system for two different sets of basis functions, namely, the eigenfunctions
of the biharmonic operator and specialized Legendre polynomials.
}

\begin{document}

\section{Introduction}
Some ideas that could go here
\begin{itemize}
  \item Solving biharmonic with FEM requires $C^1$ elements, not so easy.
    Therefore mixed formulations are more common but this leads to larger
    systems.
  \item Why Python - suitable for large scale (Oasis, spectralDNS, etc) but ease
    of implementation together with abundance of libraries (sympy, numpy) make
    it a great tool for exploration. This is what BendPy does.
  \item Question that we are asking is which of basis is better == numerically
    stable + convergence. Simple domains. Indication of suitable preconditioner.
  \item outline
\end{itemize}

\section{The math}
Let $\mathcal{I}$ denote a nonempty interval. As we are in the following
interested in domains that have a Cartesian product structure we consider without
loss of generality $\Omega=\mathcal{I}\times\mathcal{I}$. We shall refer to this
domain as \textit{plate}. Further let $w_i, i=1, 2,\cdots, k$ be a set of curves 
$w_i=\left\{\vec{x}\in\Omega, \vec{x}=\vec{F}_i\left(s\right),
s\in\mathcal{I}\right\}$, where $\vec{F_i}$ is a smooth invertible mapping with
Jacobian $J_i>0$ that additionally satisfies $\vec{F}_i\left(s_0\right),
\vec{F}_i\left(s_1\right)\in\partial\Omega$ and
$\vec{F}_i\left(s_0\right) \neq \vec{F}_i\left(s_1\right)$ for $s_0, s_1$ the
endpoints of interval $\mathcal{I}$. These curves, and occasionally the
respected mappings are referred to as \textit{beams}. In this paper we are for the
most part concerned with straight/linear beams, that is, we consider mappings
$\vec{F}_i\left(s\right)=\vec{P}_i\tfrac{s_1-s}{s_1-s_0}+\vec{Q}_i\tfrac{s_0-s}{s_0-s_1}$
determined by pairs of mutually distinct points $\vec{P}_i, \vec{Q}_i$ which
are located on the boundary of the plate. Note that in this case
$J_i=\tfrac{\left|\vec{P}_i-\vec{Q}_i\right|}{2}$.
% you do linear so absorb jacobian into material!!! KISS
%
%
%
With these assumptions on geometry we let $V, \hat{V}_i, i=1, 2, \cdots, k$ denote
spaces of functions that map respectively the plate and the beams to real
numbers. By invertibility of $\vec{F}_i$ each function space $\hat{V}_i$ can be
associated with a function space that $V_{i}$ that maps the reference interval
$\mathcal{I}$ to real numbers. Indeed for $v\in V_i$ function
$\hat{v}=v\circ\vec{F}_i$ belongs to $\hat{V}_i$.

Let $u\in V, u_i\in V_i, i=1, 2, \cdots, k$ and consider the problem of
minimizing the Lagrangian
\[
  \mathcal{L}\left(u, u_1, u_2, \cdots, u_k\right)=
  \frac{E}{2}\displaystyle\int_{\Omega}\Delta u\,\Delta u+
  \sum_i\frac{E_i}{2}\int_{\mathcal{I}}
  \deriv{^2u_i}{s^2}\deriv{^2u_i}{s^2}J_i^{-3}
  -\displaystyle\int_{\Omega}f u
\]
subjected to $k$ constraints
$T(u)=u_i$, here $T(u)$ is a short hand for the composition of trace and the
pullback inverse.\textit{explain the constraint, trace, Sobolev?}.
We build the constraint into the Lagrangian. To this end consider spaces $Q_i$
and functions (Lagrange multipliers) $\lambda_i\in Q_i$. Moreover let
$Q=Q_0\times Q_1\times\cdots\times Q_k$ and $\vvec{\lambda}\in
Q,\vvec{\lambda}_i=\lambda_i$ and a problem
\[
  \mathcal{L}\left(u, u_1, u_2, \cdots, u_k;\vvec{\lambda}\right)=
   \frac{E}{2}\displaystyle\int_{\Omega}\Delta u\,\Delta u+
  \sum_i\frac{E_i}{2}\int_{\mathcal{I}}
  \deriv{^2u_i}{s^2}\deriv{^2u_i}{s^2}J_i^{-3}
  -\displaystyle\int_{\Omega}f u - \sum_i\int_{\mathcal{I}}\left(u-u_i\right)\lambda_i J_i
\]
The necessary condition for extrema are then
\[
  \begin{aligned}
    E\displaystyle\int_{\Omega}\Delta u\,\Delta v- \sum_i\int_{\mathcal{I}}v\lambda_i J_i
    &=\displaystyle\int_{\Omega}f v\quad\forall v\in V& \\
  E_i\displaystyle\int_{\mathcal{I}} \deriv{^2u_i}{s^2}\deriv{^2v_i}{s^2}J_i^3 +
  \int_{\mathcal{I}} u_i \lambda_i J_i &= 0\quad\forall v_i\in V_i&\\
  \int_{\mathcal{I}}\left(u_i-u\right)\mu_i J_i &= 0\quad\forall \mu_i\in Q_i&\\
  \end{aligned}
\]
Under additional regularity we can make Euler Lagrange equations
\[
  \begin{aligned}
    E\Delta^2u &= f\quad\text{ in }\Omega\\
    E_i\deriv{^4 u_i}{s^4} &= 0\quad\text{ on }\mathcal{I}\\
    u &= u_i\quad\text{ on }\mathcal{I}\\
  \end{aligned}
\]
subjected to boundary conditions $u=0, \partial_nu=0$ and $u_i=0, \tderiv{u_i}{s}=0$
$u=0, \Delta u=0$ and $u_i=0, \tderiv{^2u_i}{s^2}=0$. \textit{What are these called?}
Abstract saddle $V:=V\times V_1, \times V_2, \cdots, \times V_k$,
$V\ni\vvec{u}=\left(u, u_1, u_2, \cdots, u_k\right)$ and define bilinear form
$a:V\times V\mapsto\mathbb{R}$ as
\[
  \ainner{\vvec{u}}{\vvec{v}} = 
  E\displaystyle\int_{\Omega}\Delta u\,\Delta+
  \sum_iE_i\displaystyle\int_{\mathcal{I}} \deriv{^2u_i}{s^2}\deriv{^2v_i}{s^2}J_i^3
\]
$b:V\times Q\mapsto\mathbb{R}$ as
\[
  \binner{\vvec{v}}{\vvec{\lambda}} = 
  \int_{\mathcal{I}}\left(v_i-v\right)\lambda_i J_i
\]
Finally linear form
$L:V\mapsto\mathbb{R}$ as
\[
  \displaystyle\int_{\Omega}f v
\]
%FIXME it is much better to use V_0 and then V...
% Saddle
Then the problem becomes simply: Find $\vvec{u}\in V, \vvec{\lambda}\in Q$ such
that $\ainner{u}{v}+\binner{v}{\lambda}+\binner{u}{\mu}=\Linner{v}$ for all 
$\vvec{v}\in V, \vvec{\mu}\in Q$. You have babuska theory that gives you
continuous existence. Don't go there... Discrete mention Stokes taylor hood, or
the compatibility condition from spectral.
% Galerkin
The finite dimensional approximation of $V$ is $\Vh$. Here $\mathbf{n}$ is a
multiindex of length $k+1$ and $\mathbf{n}_i, i=0, 1,\cdots, k$ denotes dimension
of the $i$-th component of $\Vh$ which is spanned by functions $\phi^i_j,
i=1,2,\cdots\mathbf{n}_i$. Note that $\phi^0_j$ are defined over $\Omega$ while
the rest have the reference interval as their domain. Similarly, we denote $\Qh$ 
the finite dimensional approximation of $Q$ is . Here $\mathbf{m}$ is a multiindex
of length $k$ and $\mathbf{m}_i, i=1, 2, \cdots, k$ denotes dimension of the 
$i$-th component of $\Qh$ spanned by functions $\psi^i_j, j=1,
2\cdots,\mathbf{m}_i$.
% Matrices
Let $n, m$ be respectively the sizes of multiindices $\mathbf{n}, \mathbf{m}$.
The abstract saddle point problem considered on the discerete subspaces
translates into the linear system: Find $\mathbf{U}\in\mathbb{R}^n,
\mathbf{P}\in\mathbb{R}^m$ such that
\[
    \begin{bmatrix}
      \mathbb{A} & \mathbb{B} \\
      \mathbb{B}^{\text{T}} & 0
    \end{bmatrix}
    \,
    \begin{bmatrix}
      \mathbf{U} \\
      \mathbf{P}
    \end{bmatrix}
    =
    \begin{bmatrix}
      \mathbf{b}\\
      0
    \end{bmatrix}.
\]
$\mathbf{b}\in\mathbb{R}^n$ has most entries zero except the first $\mathbf{n}_0$
entries which take the value $\inner{\phi^0_j}{f}$. The matrix
$\Amat\in\mathbb{R}^{n\times n}$ is block diagonal consisting of $k+1$
submatrices
\[
    \mathbb{A}=
    \begin{bmatrix}
      \mathbb{A}^0  &   &  &\\
                    & \mathbb{A}^1 &  &\\
                    &   &   \ddots    & \\
                    &   &   & \mathbb{A}^k\\
    \end{bmatrix}
\]
with $\mathbb{A}^0_{i, j}=E\inner{\Delta \phi^0_i}{\Delta\phi^0_j}$ and
$\mathbb{A}^r_{i, j}=E_r\displaystyle\int_{\mathcal{I}}
\deriv{^2\phi^r_i}{s^2}\deriv{^2\phi^r_j}{s^2}J_r^3$. Matrix
$\Bmat\in\mathbb{R}^{n\times m}$
\[
    \mathbb{B}=
    \begin{bmatrix}
      \mathbb{C}^1 & \mathbb{C}^2 & \cdots & \cdots & \mathbb{C}^k\\
      \mathbb{M}^1   &        0       & \cdots & \cdots &        0      \\
           0         & \mathbb{M}^2   &    0   & \cdots &   \vdots      \\
         \vdots      &       0        & \ddots & \ddots &   \vdots      \\
         \vdots      &     \vdots      & \ddots & \ddots &       0      \\
      0         &  \cdots        & \cdots & 0      &   \mathbb{M}^k     \\
    \end{bmatrix}
\]
where $\mathbb{C}^r_{ij}=\int_{\mathcal{I}}\phi^0_j\psi^r_i$ and
$\mathbb{M}^r_{ij}=\int_{\mathcal{I}}\phi^r_j\psi^r_i$, so the latter defines
a mass matrix between spaces.
Eigenvalue problems for wellposedness/stability. Programming needs assembler
so just take into accont the multiindex to get the position and size of the
block. What are blocks: we have mass matrix and the 1d 2d bending matrices.
Simplifies if $\phi^i=\psi^i$.

% Discussion on properties of basis. Some speedup with FFT
\section{Basis}
% For each basis how it is defined. How does A look, convergence properties for
% biharmonic problem(What happens if discontinuity). How does A scale. Comment
% on fast assembly
\subsection{Shen basis}
Described in shen. Legendre polynomials - we use for clamped boundary
conditions. Confirm spectral convergence in 1d and 2d. In 1d the biharmonic is
diagonal hence Ak matrices. But A0 is ... . Mass matrix is penta diagonal. How
does 2d bending matrix scale. Probably mention the tensor product. Mention
forward legendre transform.
\subsection{Eigen basis}
Consider eigenvalue problem. Find functions - spectral decomposition. Action of
Galerkin method with the basis. Convergence 1d, 2d. Matrices. Solution. If you
have time it would be really nice to solve 1d solution with rhs that has
different smoothness so that we see how it effects the rate in both cases.

\section{Eigenvalues}
When the basis are explained. We show kappa of both matrices as matrix
increases. Preconditioning - suggest space for the multiplier.

\section{Showcase}
Will not have rates but we can show same domain multiple beans different bcs. If
the nonlinearity is included.... If the precond works at least in one case solve
the problem iteratively.
% Preconditioning, uzawa, block lu

\section{Conclusions and extensions}
Combine different families of functions...

\end{document}
